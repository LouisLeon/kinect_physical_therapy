\documentclass[onecolumn, draftclsnofoot,10pt, compsoc]{IEEEtran}
\usepackage{graphicx}
\usepackage{pgfgantt}
\usepackage{url}
\usepackage{setspace}
\usepackage{tabu}
\usepackage{geometry}
\usepackage{listings} % code listings
\usepackage{color} % colors for code listings
\usepackage{float}
\usepackage{caption}
\graphicspath{{images/}}
\geometry{textheight=9.5in, textwidth=7in, margin=0.75in}

\definecolor{mygreen}{rgb}{0,0.6,0}
\definecolor{mygray}{rgb}{0.5,0.5,0.5}
\definecolor{mymauve}{rgb}{0.58,0,0.82}

\lstset{ %
  backgroundcolor=\color{white},   % choose the background color
  basicstyle=\footnotesize,        % size of fonts used for the code
  breaklines=true,                 % automatic line breaking only at whitespace
  captionpos=b,                    % sets the caption-position to bottom
  commentstyle=\color{mygreen},    % comment style
  escapeinside={\%*}{*)},          % if you want to add LaTeX within your code
  keywordstyle=\color{blue},       % keyword style
  stringstyle=\color{mymauve},     % string literal style
}
\lstdefinestyle{customc}{
  belowcaptionskip=1\baselineskip,
  breaklines=true,
  frame=L,
  xleftmargin=\parindent,
  language=C,
  showstringspaces=false,
  basicstyle=\footnotesize\ttfamily,
  keywordstyle=\bfseries\color{mygreen},
  commentstyle=\itshape\color{mygreen},
  identifierstyle=\color{blue},
  stringstyle=\color{orange},
}

% 1. Fill in these details
\def \CapstoneTeamName{     TeamName}
\def \CapstoneTeamNumber{       24}
\def \GroupMemberOne{            Ciin S. Dim}
\def \GroupMemberTwo{           Louis Leon}
\def \GroupMemberThree{         Karl Popper}
\def \CapstoneProjectName{      Kinect Based Virtual Therapy Solution}
\def \CapstoneSponsorCompany{   OSU Healthcare Systems Engineering Lab}
\def \CapstoneSponsorPerson{        Mehmet Serdar Kilinc}

% 2. Uncomment the appropriate line below so that the document type works
\def \DocType{      %Problem Statement
                %Requirements Document
                %Technology Review
                %Design Document
                Progress Report
                }
            
\newcommand{\NameSigPair}[1]{\par
\makebox[2.75in][r]{#1} \hfil   \makebox[3.25in]{\makebox[2.25in]{\hrulefill} \hfill        \makebox[.75in]{\hrulefill}}
\par\vspace{-12pt} \textit{\tiny\noindent
\makebox[2.75in]{} \hfil        \makebox[3.25in]{\makebox[2.25in][r]{Signature} \hfill  \makebox[.75in][r]{Date}}}}
% 3. If the document is not to be signed, uncomment the RENEWcommand below
\renewcommand{\NameSigPair}[1]{#1}

%%%%%%%%%%%%%%%%%%%%%%%%%%%%%%%%%%%%%%%
\begin{document}
\begin{titlepage}
    \pagenumbering{gobble}
    \begin{singlespace}
        %\includegraphics[height=4cm]{coe_v_spot1}
        \hfill 
        % 4. If you have a logo, use this includegraphics command to put it on the coversheet.
        %\includegraphics[height=4cm]{CompanyLogo}   
        \par\vspace{.2in}
        \centering
        \scshape{
            \huge CS Capstone\DocType \par
            {\large Spring Term}\par
            {\large\today}\par
            \vspace{.5in}
            \textbf{\Huge\CapstoneProjectName}\par
            \vfill
            {\large Prepared for}\par
            \Huge \CapstoneSponsorCompany\par
            \vspace{5pt}
            {\Large\NameSigPair{\CapstoneSponsorPerson}\par}
            {\large Prepared by }\par
            Group\CapstoneTeamNumber\par
            % 5. comment out the line below this one if you do not wish to name your team
            %\CapstoneTeamName\par 
            \vspace{5pt}
            {\Large
                \NameSigPair{\GroupMemberOne}\par
                \NameSigPair{\GroupMemberTwo}\par
            }
            \vspace{20pt}
        }
        \begin{abstract}
        % 6. Fill in your abstract    
        The purpose of this document is to summarize the progress made towards this project over the first half of Spring Term. The document includes the project purpose, goals, current project state, problems impeding progress and solutions, and remaining tasks.
    \end{abstract}     
    \end{singlespace}
\end{titlepage}
\newpage
\pagenumbering{arabic}
\tableofcontents
% 7. uncomment this (if applicable). Consider adding a page break.
\listoffigures
%\listoftables
\clearpage

% 8. now you write!
\section{Purpose}
The purpose of the Kinect Based Physical Therapy Solution project is to provide a solution for physical therapy patients diagnosed with Parkinson's disease to perform in-home therapy exercises. This solution will not only allow for an interactive way of completing a patient's required home therapy but will provide a way for their physical therapist to track their progress and monitor their exercises.

\section{Goals}
Our project will have a simple UI that can be easily navigated by a user with Parkinson's Disease. From the UI, the user will be able to select the option that allows them to select and do the available exercises. Our goal is to have two different exercises available to the user to choose from. One of our stretch goals is to be able to have a physical therapist prescribe exercises and specify frequency. The program will guide the user through these exercises using text and verbal instructions. One of our stretch goals is to implement visual (graphical) cues (over-laying their body on the camera feed) to the user to guide them through exercises. As the user performs exercises, their node data will be collected to be sent to their physical therapist as a .csv file, and it will be used for report generation. We will define the exercise's correct movements and compare them to the user's node data. This is how we will analyze the data to determine user accuracy for report generation. Another option the user will be able to select is report generation. This will display the user's performance data in graphs and charts, showing their progress with the exercises over time. 

\section{Current Project State}
To begin addressing the bugs in our project from the previous term, we reconstructed the user interface to have a more simple design. This new version contains all of the same modules, but with less baggage and latency. The home page of the user interface has the following pages: Exercises, Reports, and Data Collection Settings. Selecting Reports leads the user to a page displaying graphs of their exercise performance. The Data Collection Settings page is where the start, stop, and frequency of data collection can be customized.

The Exercises menu option leads the user to a Exercise Options page where the options Arm Raises, Walking, and Sitting and Standing are displayed on large buttons. Selecting one of these will begin the corresponding exercise. When beginning an exercise, the user can enter how many repetitions they will do based on their therapist's prescription. Once an exercise is complete, the user is prompted to save their node data in the preferred location. Selecting Back will navigate them back to the home page. 

\section{Problems Impeding Progress and Solutions}
\subsection{Body Tracking Latency}
When navigating to and from different exercises, our program was extremely slow. Opening the first exercise showed normal body tracking behavior. When exiting that page and opening another exercise, the body tracking was visibly lagging. At this state, the pages were of the type UserControl, and used a navigation method that didn't have any garbage collection. Because of this, the body tracking of each opened page would continue to run in the background, causing the display to lag behind.

We solved this problem by reconstructing the whole project. Starting with a blank WPF project in Visual Studio, we rebuilt each page one by one using only the necessary pieces. In this new project, each page is of the type Page and is displayed inside a MainWindow. This new version uses NavigationService to control navigation between pages. This ultimately solved the latency issue because NavigationService has its own garbage collection built into the data type. When navigating to a page, an instance of that page is created. When navigating away, it is destroyed, eliminating the problem of unnecessary background processes.

\subsection{Data Collection Customization}
When implementing data collection, we realized that the library we are using allows very high level access. It wouldn't allow us to customize the format of the file or, more importantly, customize the frequency of data collection. This was concerning to us because it is one of the required features of our product and one that our client was most interested in. Without being able to customize data collection, it would no longer be beneficial for the research conducted by our client.

Luckily, our solution was simple. Since we were able to access the source code of this library, we went ahead and changed the functions to serve our purposes. We added a column to show a timestamp of the row of data as well as changing how often data was written to the file.

\subsection{User Interface}
After our initial testing and design of the user interface, our client requested that we change some of the user interface structure and looks. Some issues with the user interface involved some of the graphical elements of the program blocking the viewing window of the user. We were using a template layout that came pre-packaged with the Kinect SDK which allowed for a main navigation screen with tiles that acted as links to other pages within the program\cite{KinectDevelop}.

Each page contained a navigation section and title near the upper-half of the screen. It allowed the user to press a \textit{back} button to go to the main screen. This element of the software was rather large which limited the viewing space for the main activities of our application. The user interface template itself was slightly bulky and bland looking so we decided to revamp the structure and looks of the entire interface. Due to the target audience of our software, we needed to make the navigation more accessible and easy to use. We previously had a square \textit{back} button which occupied a smaller area of the top-left corner of the screen. It was not as easy to navigate back and forth between pages this way so we made a change to the button and made it less-obtrusive by having it be the shape of a long vertical bar that spanned from the top of the page all the way to the bottom. By placing this new \textit{back} button to the far left of the user's view, it made navigating far more intuitive and allowed for the main program to fill in more of the previous blank space.

Our solution included a new look and the addition of a higher contrast color palette for better visibility. The main navigation screen housed tiles which link to pages similar to the previous design, however the tiles are cleaner looking and more smooth around the edges. This gives the program a more friendly and approachable interface which will hopefully make the usage of our application seem less intimidating and more inviting. The new page format is consistent across all of the pages in our application. The user is also still able to use hand movements and gestures to navigate between pages.

\subsection{Report Generation}
Some of the difficulties we faced while developing the report generation portion of this project were design and requirements related. We knew that we wanted a page in the software which had the purpose of generating reports based on the exercises that the user completes. We also knew that the page had to read in some arbitrary amount of data that was stored somewhere locally that was related to the user. We had the issue of deciding where and how to store this data locally and how to access it for the purposes of generating some visual report for the client and user to see. Initially, our data was being collected during each of the available exercises. The data would be written into a .txt file during the exercise session and written to that file in an unspecified order. The issue came with being able to read the file successfully enough and parse the data needed to generate some form of usable chart. 

We were also unsure of which data would be useful to the user and how to present it in an easily digestible format. Our solution was to change the manner in which the user and report data was collected during the exercise session by switching the format to a .csv file or comma separated values. This allows us to continually write to a file that corresponds with the current exercise. This continuous record can be dynamically plotted within the program's view using a library called \textit{OxyPlot}\cite{OxyPlot}. 

Within our code, we made some plot model classes and .csv reading classes that read in the user exercise data and restructure it to match the required format for the plotting library. This also allows a user to choose which exercise they wish to see their generated reports with. The reports draw graphs that display data such as the date of their exercise and the time it took to complete a specific exercise. A patient or their physical therapist can monitor performance over time or see the limb angles achieved during a session\cite{csvFile}.

\section{Remaining Tasks}
\subsection{Node Clipping}
One of our remaining tasks for this project is to fix an issue with our visual skeleton drawings. Our software allows the user to see themselves along with an overlaid image of several body joints and nodes. There are also some drawn lines that are associated with the user's limbs and spine. These shapes are all drawn simultaneously once the Kinect Sensor determines there to be a human body in front of the camera. 

There may be some times where the Kinect Sensor is pointed at an angle such that it does not see the user's lower body within its visual scope. This leads to the associated lower body lines and joint shapes to be clipped and appear non-connecting. This issue disappears once the user's entire body enters the frame and is recognized by the sensor. 

Unfortunately, this behavior is caused by the library we are using to help us draw all of the main shapes needed as well as some angle visualizations. It makes the process lighter and simpler than using the built-in Kinect methods which requires more setup in the code portions. One solution is to resort back to the built-in methods only to draw the overlaid body lines and joints while using the installed Vitruvius library to do the rest\cite{Vitruvius}.
\section{Relevant Project Information}


\section{Conclusion}
At this point, we have completed the features outlined in our Requirements document and Design Document. From now until the Expo, we will be fixing minor bugs and making changes based on feedback we got from our client. Reconstructing the user interface addressed his concerns about latency and usability. The new UI has solved the latency issue, and the buttons utilize more of the screen space. We have addressed, and will continue to address, each bug one at a time.

\newpage
\bibliographystyle{ieeetr}
\bibliography{SpringProgressReport}

\end{document}
