\documentclass[onecolumn, draftclsnofoot,10pt, compsoc]{IEEEtran}
\usepackage{graphicx}
\usepackage{url}
\usepackage{setspace}

\usepackage{geometry}
\geometry{textheight=9.5in, textwidth=7in}

% 1. Fill in these details
\def \CapstoneTeamName{     The Cleverly Named Team}
\def \CapstoneTeamNumber{       24}
\def \GroupMemberOne{            Ciin S. Dim}
\def \GroupMemberTwo{           Louis Leon}
\def \GroupMemberThree{         Karl Popper}
\def \CapstoneProjectName{      Kinect Based Virtual Therapy Solution}
\def \CapstoneSponsorCompany{   OSU Healthcare Systems Engineering Lab}
\def \CapstoneSponsorPerson{        Mehmet Serdar Kilinc}

% 2. Uncomment the appropriate line below so that the document type works
\def \DocType{      Problem Statement
                %Requirements Document
                %Technology Review
                %Design Document
                %Progress Report
                }
            
\newcommand{\NameSigPair}[1]{\par
\makebox[2.75in][r]{#1} \hfil   \makebox[3.25in]{\makebox[2.25in]{\hrulefill} \hfill        \makebox[.75in]{\hrulefill}}
\par\vspace{-12pt} \textit{\tiny\noindent
\makebox[2.75in]{} \hfil        \makebox[3.25in]{\makebox[2.25in][r]{Signature} \hfill  \makebox[.75in][r]{Date}}}}
% 3. If the document is not to be signed, uncomment the RENEWcommand below
\renewcommand{\NameSigPair}[1]{#1}

%%%%%%%%%%%%%%%%%%%%%%%%%%%%%%%%%%%%%%%
\begin{document}
\begin{titlepage}
    \pagenumbering{gobble}
    \begin{singlespace}
        %\includegraphics[height=4cm]{coe_v_spot1}
        \hfill 
        % 4. If you have a logo, use this includegraphics command to put it on the coversheet.
        %\includegraphics[height=4cm]{CompanyLogo}   
        \par\vspace{.2in}
        \centering
        \scshape{
            \huge CS Capstone\DocType \par
            {\large Fall Term}\par
            {\large\today}\par
            \vspace{.5in}
            \textbf{\Huge\CapstoneProjectName}\par
            \vfill
            {\large Prepared for}\par
            \Huge \CapstoneSponsorCompany\par
            \vspace{5pt}
            {\Large\NameSigPair{\CapstoneSponsorPerson}\par}
            {\large Prepared by }\par
            Group\CapstoneTeamNumber\par
            % 5. comment out the line below this one if you do not wish to name your team
            %\CapstoneTeamName\par 
            \vspace{5pt}
            {\Large
                \NameSigPair{\GroupMemberOne}\par
                \NameSigPair{\GroupMemberTwo}\par
            }
            \vspace{20pt}
        }
        \begin{abstract}
        % 6. Fill in your abstract    
            The purpose of this document is to define and describe a possible solution for physical therapists to utilize when monitoring a patient's prescribed therapeutic movement set. The solution involves the use of a Kinect sensor to track a patient's movements when performing exercises. The data that the sensor records will be stored and sent to their physical therapist to allow them to monitor their patient's progress. The task is to develop software that includes an interface for patients and physical therapists to interact with. Pre-defined exercises will be implemented in the software and compared against a patient's movements to determine the accuracy of the therapy. The project will be completed once a working prototype is prepared and the clients' requirements are satisfied. The document is structured into three sections which provide a high-level description of the problem, solution, and performance metrics.
        \end{abstract}     
    \end{singlespace}
\end{titlepage}
\newpage
\pagenumbering{arabic}
\tableofcontents
% 7. uncomment this (if applicable). Consider adding a page break.
%\listoffigures
%\listoftables
\clearpage

% 8. now you write!
\section{Problem Definition and Description}
    We are working with Mehmet Serdar Kilinc, a research associate, and Jose Castro, a PhD student, who conduct research at the OSU Healthcare Systems Engineering Lab. The problem is some patients have a difficult time accessing physical therapy clinics due to a physical disability that inhibits travel. The goal of the project is to develop a sensor-based solution, using the Xbox Kinect, that brings physical therapy sessions to the patients. This has the potential to reduce the number of in-person visits and long-distance travel that burdens patients. There is also an opportunity to provide more objective frequent motor symptom monitoring.\par

    This virtual therapy solution will guide patients through exercises by giving spoken and written instructions and sense whether they are doing the motions correctly. Through the use of a Kinect sensor which tracks the human body movement, we hope to develop some software that stores and tracks the movements of the patient. That information can then be sent to their physical therapist who can analyze and further prescribe any exercises they deem necessary for their patient. Like any other prescription, the patient may have to do certain exercises a specified number of times correctly. The physical therapist can verify if their patients have been performing their prescribed exercises correctly given the sensor data and generated reports. The sensor will be more accurate than a physical therapist's eyes making it more likely to detect early stages of Parkinson's disease.\par

    When working with patients with a disability such as Parkinson's disease, the software we develop may be used to track their symptoms over time by measuring common symptoms and movements of their body. A physical therapist can then analyze the data over time to see if their condition has worsened. A tool such as this may also be very useful when detecting early signs of developing Parkinson's disease. The physical therapist should be able to generate reports and customize exercises for each individual patient using this tool. An interface needs to be implemented which allows patients to interact with the software and receive instructions as well as feedback about their assigned movement sets. The software must be accurate and simply interactive in order for the patient to receive proper treatment.  

\section{Proposed Solution}
The first step in our proposed solution towards this project will be to become familiar with the Kinect sensor and its various capabilities. This knowledge will be essential for programming the back end of the software. As developers, we have the option of specifying the manner in which the sensor can track a human body. This includes tracking human skeleton and joint movements and collecting data from the accelerometer. The Kinect system uses 20-25 nodes to track human movements. Each node, at each moment in time, has a 3D coordinate. The coordinates of each node will be recorded at specified intervals of time. we will program pre-defined exercises such as repeated arm movements that a patient might have to perform. The program will instruct the patient to raise their arm in a specific way and then measure their range of motion and speed. The Kinect sensor also has built in audio processing that may prove extremely helpful with patients who have limited range of motion or difficulty standing. \par

Using their node data, we can determine how accurate the patient's movements are to the given exercise they follow. This will be done by comparing the node positions of the patient's recorded data to the node positions of the pre-defined exercises. Our solution also has the potential to let the user know whether or not they are correctly completing the exercises through an on-screen interface.The patient may be asked to try again for a certain amount of times before we can come to the conclusion that the patient's range of motion has been limited. \par

To create the user interface, start and stop parameters, and exercises, we will be using the Kinect SDK. To define the exercises, we will be consulting physical therapists from Dixon Recreational Center and/or the Samaritan Physical Rehabilitation Clinic. By learning the important conditions of a proper physical movement from the physical therapists, we will then be able to define these movements in the software using node coordinates. This may also require the physical therapists to do several example movements in front of the sensor to get an idea of the node positions of a correct movement. \par

The other component of this project is data collection. The data collected from each node throughout a session will be saved to an excel file in which each column is a node containing its position, and each row is a point in time. In addition to the raw position data in the excel file, we will also include meta data such as the timestamp of recorded data and duration of recorded data. As the sensor collects data, we will process that data and store it in a file where it can be sent to a physical therapist. This will allow the physical therapist to look at the data and prescribe or modify exercises, as well as diagnose early signs of any potential disability.

\section{Performance Metrics}
Since this is primarily a research project, Mehmet and Jose had some trouble coming up with metrics to define a complete project. In general, they both agreed that the outcome of a research project is unpredictable. Unlike a project like developing a specific product for industry, research projects have no definite timeline or answers. With this in mind, we concluded that our project is “complete” if it has the following minimum features:
\begin{itemize}
    \item \textbf{A working interface for patients to interact with the program:} A patient should to be able to receive and read on-screen instructions for physical therapy exercises. The program will be able to recognize users and accurately track their body movements as well as provide spoken instructions to patients if needed. The solution should identify a body and map nodes to it as well as contain an interface that is usable by someone who doesn't have any knowledge of the development process.
    \item \textbf{Setting the duration of a data collection period:} This will trim the data in a way that shows us windows of activity to assess, which is more manageable and easier to comprehend than the constant raw influx of data.
    \item \textbf{Node angle analysis:} The sensor will be able to look at the nodes being tracked and the software will be able to calculate at what angle a patient is moving a specific limb. This can give a physical therapist more insight on how much range of motion a patient may have in different areas of their body. 
    \item \textbf{Start and stop rules:} These are the conditions, when met, that determine when to start an exercise and when to stop. The software needs to know when to start and stop paying attention to how the user is moving compared to the pre-defined correct movements. This is also part of the user experience. The start and stop rules need to be somewhat intuitive to the user and easy enough to do. This could be based on distance/position from the sensor, movement, or pose.
    \item \textbf{Timestamps of saved data:} The data that is collected will need timestamps as a way to keep track of patients' progress. Data is taken multiple times a second, but there needs to be a way to view intervals such as exercise sessions.
    \item \textbf{Setting the time and frequency of data saving:} This allows the user and/or the physical therapist to choose how often the patients' data will be saved, and when the data should be saved.
\end{itemize}
In addition to these minimum data collection features, we would like to have at least one or two pre-defined exercises completed to be able to demo at the expo.


\end{document}