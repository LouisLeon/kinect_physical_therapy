\documentclass[onecolumn, draftclsnofoot,10pt, compsoc]{IEEEtran}
\usepackage{graphicx}
\usepackage{pgfgantt}
\usepackage{url}
\usepackage{setspace}
\usepackage{tabu}
\usepackage{geometry}
\geometry{textheight=9.5in, textwidth=7in, margin=0.75in}

% 1. Fill in these details
\def \CapstoneTeamName{     TeamName}
\def \CapstoneTeamNumber{       24}
\def \GroupMemberOne{            Ciin S. Dim}
\def \GroupMemberTwo{           Louis Leon}
\def \GroupMemberThree{         Karl Popper}
\def \CapstoneProjectName{      Kinect Based Virtual Therapy Solution}
\def \CapstoneSponsorCompany{   OSU Healthcare Systems Engineering Lab}
\def \CapstoneSponsorPerson{        Mehmet Serdar Kilinc}

% 2. Uncomment the appropriate line below so that the document type works
\def \DocType{      %Problem Statement
                %Requirements Document
                %Technology Review
                Design Document
                %Progress Report
                }
            
\newcommand{\NameSigPair}[1]{\par
\makebox[2.75in][r]{#1} \hfil   \makebox[3.25in]{\makebox[2.25in]{\hrulefill} \hfill        \makebox[.75in]{\hrulefill}}
\par\vspace{-12pt} \textit{\tiny\noindent
\makebox[2.75in]{} \hfil        \makebox[3.25in]{\makebox[2.25in][r]{Signature} \hfill  \makebox[.75in][r]{Date}}}}
% 3. If the document is not to be signed, uncomment the RENEWcommand below
\renewcommand{\NameSigPair}[1]{#1}

%%%%%%%%%%%%%%%%%%%%%%%%%%%%%%%%%%%%%%%
\begin{document}
\begin{titlepage}
    \pagenumbering{gobble}
    \begin{singlespace}
        %\includegraphics[height=4cm]{coe_v_spot1}
        \hfill 
        % 4. If you have a logo, use this includegraphics command to put it on the coversheet.
        %\includegraphics[height=4cm]{CompanyLogo}   
        \par\vspace{.2in}
        \centering
        \scshape{
            \huge CS Capstone\DocType \par
            {\large Fall Term}\par
            {\large\today}\par
            \vspace{.5in}
            \textbf{\Huge\CapstoneProjectName}\par
            \vfill
            {\large Prepared for}\par
            \Huge \CapstoneSponsorCompany\par
            \vspace{5pt}
            {\Large\NameSigPair{\CapstoneSponsorPerson}\par}
            {\large Prepared by }\par
            Group\CapstoneTeamNumber\par
            % 5. comment out the line below this one if you do not wish to name your team
            %\CapstoneTeamName\par 
            \vspace{5pt}
            {\Large
                \NameSigPair{\GroupMemberOne}\par
                \NameSigPair{\GroupMemberTwo}\par
            }
            \vspace{20pt}
        }
        \begin{abstract}
        % 6. Fill in your abstract    
            The purpose of this document is to define and describe a possible solution for physical therapists to utilize when monitoring a patient's prescribed therapeutic movement set. The solution involves the use of a Kinect sensor to track a patient's movements when performing exercises. The data that the sensor records will be stored and sent to their physical therapist to allow them to monitor their patient's progress. The task is to develop software that includes an interface for patients and physical therapists to interact with. Pre-defined exercises will be implemented in the software and compared against a patient's movements to determine the accuracy of the therapy. The project will be completed once a working prototype is prepared and the clients' requirements are satisfied. The document is structured into three sections which provide a high-level description of the problem, solution, and performance 
metrics.
        \end{abstract}     
    \end{singlespace}
\end{titlepage}
\newpage
\pagenumbering{arabic}
\tableofcontents
% 7. uncomment this (if applicable). Consider adding a page break.
%\listoffigures
%\listoftables
\clearpage

% 8. now you write!
\section{Introduction}
\subsection{Purpose}
The purpose of this document is to describe the details of major components of this project. Each component will be depicted from various design viewpoints. The details outlined in each component will define the standards of completion. This document can be used by the client to confirm what will be implemented. The developers will use this document as a guide to implementing the project and for testing the final product to ensure it works as intended. The developers and client can then use this document to determine whether the final product meets the standards defined.

\subsection{Scope}
The components described in this document are the body tracking sensor, organization of the user interface, exercise recognition, data collection, report generation, and data visualization. 

\subsection{Summary}
The \textit{Kinect Physical Therapy Solution} uses a Kinect body tracking sensor to collect the data of a user. The user can specify which exercise they want to begin using the user interface and menu system. During the exercise, the user's skeletal data is captured and processed by the data collector and exercise recognizer, respectively. A report is generated for the user and acts as a progress chart to see their improvements over time. The data for each exercise session is captured and parsed into a .csv file with the intention of storing locally and sending to the patient's physical therapist. 

\subsection{Glossary}
        \begin{tabu} to \hsize {|X|X[2,l]|}
        \hline
        \textbf{Term} & \textbf{Definition}\\
        \hline
        .csv File & Stands for "comma-separated values". A file format that is used to store tabular data, such as a spreadsheet or database. They may be imported/exported into different programs that store data in tables\cite{csvFile}.\\
        \hline
        Node & In this context, a node represents a point along the body of the user. Typically associated with a joint in the skeletal system\cite{KinectDevelop}.\\
        \hline
        PC & Personal Computer\\
        \hline
        Kinect For Windows SDK & A software development kit used to program the Kinect sensor\cite{KinectDevelop}.\\
        \hline
\end{tabu}
\subsection{References}
\bibliographystyle{ieeetr}
\bibliography{DesignDocument}

\newpage
\subsection{Overview}
This document will provide an overall description of our project and expected final 
product. This description will include product perspective, product functions, user 
characteristics, constraints, and assumptions and dependencies. Following the overall 
description are the specific requirements that outline our product's functionality as 
well as a project schedule.

\subsection{Schedule}
%Gantt Chart Goes HERE
\begin{ganttchart}{1}{22}
  \gantttitle{TITLE}{22} \\
  \gantttitle[]{Fall}{5}
  \gantttitle[]{Winter}{10}
  \gantttitle[]{Spring}{7}\\
  \gantttitle{6}{1}
  \gantttitle{7}{1}
  \gantttitle{8}{1}
  \gantttitle{9}{1}
  \gantttitle{10}{1} 
  \gantttitle{1}{1} 
  \gantttitle{2}{1}
  \gantttitle{3}{1}
  \gantttitle{4}{1}
  \gantttitle{5}{1}    
  \gantttitle{6}{1}
  \gantttitle{7}{1}
  \gantttitle{8}{1}
  \gantttitle{9}{1}
  \gantttitle{10}{1} 
  \gantttitle{1}{1} 
  \gantttitle{2}{1}
  \gantttitle{3}{1}
  \gantttitle{4}{1}
  \gantttitle{5}{1}    
  \gantttitle{6}{1}
  \gantttitle{7}{1}\\

  \ganttbar{Define start and stop rules}{1}{3}\\
  \ganttbar[inline=false]{Research existing kinect solutions}{4}{5}\\

  \ganttbar{Export formatted data to .csv}{6}{6}\\
  \ganttbar{Set frequency and duration of data collection}{7}{7}\\
  \ganttbar{Implement node angle analysis}{8}{9}\\
  \ganttbar{Define correct movements}{10}{11}\\
  \ganttbar{Compare user vs. correct movements}{12}{13}\\
  \ganttbar{Implement feedback to user}{14}{15}\\

  \ganttbar{Landing page}{16}{16}\\
  \ganttbar{Menu}{17}{17}\\
  \ganttbar{Written instructions}{18}{19}\\
  \ganttbar{Testing and preparing for expo}{20}{22}\\
\end{ganttchart}

\newpage
%Start the viewpoints here
\section{Body Tracking Sensor}
\subsection{Context}
The use of the Microsoft Kinect body tracking sensor will be essential for the tracking of users and their exercises. The sensor will act as the main interface between the user and the software.  

\subsection{Dependencies}
A Windows 8 or higher operating system is needed to run the program and support our chosen model of the Kinect sensor (v2). The operating system hosts and runs the program at the users request which will then load the starting menu for exercise selection or data capture configuration. The sensor also requires the Microsoft .NET Framework 4.5 for development \cite{KinectDevelop}. 

\subsection{Composition}
The Kinect sensor connects to a PC though USB 3.0 with the addition of standard power input. Development does not directly require the sensor as it functions mainly for a source of input for the software. A Windows device driver will be controlling the device during operation. The sensor is able to track up to six bodies through the use of 25-point skeletal mapping and supports open-hand or closed-hand gesture recognition. The sensor also has a built-in microphone to detect voice commands which may be implemented as a stretch goal for our project which would let users interact with the useron interface and menu systems \cite{KinectConstraints}. 

\subsection{Interactions}
The sensor interacts with the software via the \textit{Kinect for Windows} SDK. User skeletal data captured by the sensor is handled and processed by the exercise recognition modules in the software. All of the data being captures is collected via the data collection modules which parse the data out into a formatted .csv file. 

\section{User Interface}
\subsection{Context}
The main user interface for this project will be organized in a very simple manner. This will be in an effort to minimize user confusion and make the overall experience more intuitive. The user interface helps with the selection of the desired exercise and report generation. The structure and organization of the user interface will be composed of menu options and visual cues. 

\subsection{Dependencies}
Structuring of the elements for the user interface will be done through the use of Windows Presentation Foundation UI Framework (WPF). WPF provides the ability to create a user interface through Microsoft Visual Studio and supports images, 3D graphics, and video/audio \cite{WPFSamples}. 

\subsection{Composition}
The composition of the UI includes a main menu, exercise selection, settings, report generation, and data exportation. The main menu is where the user will start as they initiate the program. From the main menu, a user will have the option of starting an exercise routine. Once the exercise routine option is selected from the main menu, various exercises will appear for the user to choose from. The exercise starts once selected and on-screen instructions appear to guide the user through the required movements. Once complete, a report of their performance is generated and the exercise routine selection screen appears. The user interface should also include an overlay during the exercise portion which indicates how well the user is doing.  

\subsection{Interactions}
The user interface provides the software with specific inputs and configurations for data collection and exercise recognition. The corresponding verification will need to be active with the currently selected exercise such that the user receives correct feedback and reports are properly generated.  

\section{Exercise Recognition}
\subsection{Context}
The exercise recognition aspect of the project is what reads the output of the sensor and compares it with a baseline version of an exercise. It serves as a comparison for the user and will be helpful when indicating feedback to the user. The user does not directly interact with this piece of the software but instead follows the on-screen instructions for performing the specific exercise they selected for the current session.  

\subsection{Dependencies}
Recognition of user exercises heavily depends on the \textit{Kinect for Windows} SDK for API calls. Specifically, the skeletal and node data of the user's body is required as an input in order to begin evaluating their movements. 

\subsection{Composition}
The elements which make up exercise recognition include the detection of linear gestures or movements that follow a given path. The path is to be defined by the developer and is built into the software. There will be pre-defined movements for each of the available exercises. The following steps describe the method for recognizing linear gestures: 
\begin{itemize}
\item Check that all necessary points are in progression to the correct direction 
\item Check if necessary limbs or joints fall within a specified range on the x, y, and z axes 
\item Check if a minimum distance has been covered with the movement 
\item Check that the first and last positions were created within a given period of time 
\end{itemize}


\subsection{Interactions}
The exercise recognition module mainly receives inputs from the sensor data and directly accesses the values of the user's skeletal position in order to process and verify it. If the user is performing a movement or gesture incorrectly, a visual cue will be shown on-screen requesting that the user adjust their body while the exercise recognition continues checking until it deems the movement to be correct. User accuracy will be logged and reported in the data visualization and report generation after they finish performing their exercise. 

\section{Data Collection}
\subsection{Context}
\subsection{Dependencies}
\subsection{Composition}
\subsection{Interactions}

\section{Report Generation}
\subsection{Context}
\subsection{Dependencies}
\subsection{Composition}
\subsection{Interactions}

\section{Data Visualization}

\end{document}
